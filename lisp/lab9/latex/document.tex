\documentclass[a4paper, 12pt]{article}

%\usepackage{cmap}
\usepackage[T2A]{fontenc}
\usepackage[utf8]{inputenc}
\usepackage[english, russian]{babel}
\usepackage{graphicx}
\usepackage[top=1in, bottom=1in, left=3.2cm, right=2.6cm]{geometry}
\graphicspath{./}
\usepackage{biblatex}
\addbibresource{lib.bib}
\linespread{1.5}
\usepackage{ragged2e}
\justifying
\usepackage{listings}
\usepackage{color}


\begin{document}
	
\begin{titlepage}
	\fontsize{12pt}{12pt}\selectfont
	\begin{figure}[t!]
		\centering
		\includegraphics[scale=0.8]{bmstu}
	\end{figure}
	
	\noindent\rule{15cm}{3pt}
	\newline\newline
	\noindent 
	ФАКУЛЬТЕТ 
	\underline{«Информатика и системы управления»} \newline\newline
	
	\noindent КАФЕДРА \underline{«Программное обеспечение ЭВМ и информационные технологии»}\newline\newline\newline\newline\newline
	
	\centering {\Large Отчет по лабораторной работе № 8}
	\vspace{4mm}
	
	\centering {\Large По курсу: "Функциональное и логическое программирование"
		\vspace{8mm}	
		
		}
	\vspace{8mm}
	
	
	\begin{flushright}
		{\small	Студент:\\ Турсунов Жасурбек Рустамович \\ Группа: ИУ7-66Б
			\vspace{3mm}
			\\Преподователи: \\ Толпинская Наталья Борисовна \\ Строганов Юрий Владимирович}
	\end{flushright}
	
	\begin{center}
		\vfill
		Москва, \the\year
		~г.
	\end{center}
\end{titlepage}

\tableofcontents
\clearpage
\newpage

\textbf{Цель работы:} приобрести навыки использования рекурсии и функционалов.
\\ \hspace*{5mm} \textbf{Задачи работы:} изучить способы применения функционалов и рекурсии при обработке одноуровневых и структурированных списков.


\section*{Введение}
\addcontentsline{toc}{section}{Введение}

\hspace*{5mm} Для организации многократных вычислений в Lisp могут быть использованы функционалы - функции, которые особым образом обрабатывают свои аргументы. Функционалы - это  функции более высокого порядка, так как они в качестве своего первого аргумента принимают функциональный объект - функцию, имеющую имя(глобально определенную функцию), или функцию, не имеющею имени(локально определенную функцию). При использовании рекурсии - необходимо обеспечить эффективность работы, путем использования хвостовой рекурсии. В рекурсивных функциях могут быть использованы дополнительные функции - не в аргументах вызова, а вне них. Рекурсивные вызовы могут быть организованы как в одной ветке, так и в нескольких ветках программы. Рекурсивные функции могут вызывать друг друга. 
\clearpage
\newpage


\definecolor{codegreen}{rgb}{0,0.6,0}
\definecolor{codegray}{rgb}{0.5,0.5,0.5}
\definecolor{codepurple}{rgb}{0.58,0,0.82}
\definecolor{backcolour}{rgb}{0.95,0.95,0.92}

\lstdefinestyle{mystyle}{
	backgroundcolor=\color{backcolour},   
	commentstyle=\color{codegreen},
	keywordstyle=\color{magenta},
	numberstyle=\tiny\color{codegray},
	stringstyle=\color{codepurple},
	basicstyle=\ttfamily\footnotesize,
	breakatwhitespace=false,         
	breaklines=false,                 
	captionpos=b,                    
	keepspaces=true,                 
	numbers=left,                    
	numbersep=5pt,                  
	showspaces=false,                
	showstringspaces=false,
	showtabs=false,                  
	tabsize=4
}

\lstset{style=mystyle}

\section*{Задание 1}
\addcontentsline{toc}{section}{Задание 1}
Написать функцию, вычисляющую декартово произведение двух своих списков-аргументов.

\begin{lstlisting}[caption=Функция вычисления декартова произведения]
(defun decart (set1 set2)
	(mapcan #'(lambda (x)
		(mapcar #'(lambda (y) 
			(cons x (cons y nil))
			) set2
		)
	  ) set1
	)
)
\end{lstlisting}
\textbf{set1} и \textbf{set2} - списки-множества.

С помощью mapcar получается декартово произведение одного элемента первого множества на второе множество, с помощью mapcan получается декартово произведение всех элементов первого множества на второе.

\subsubsection*{Примеры работы:}
\begin{lstlisting}
	> (decart '(1 2) '(a b c))
	((1 A) (1 B) (1 C) (2 A) (2 B) (2 C))
	> (decart '(1) '(A))
	((1 A))
	> (decart '(1) nil)
	NIL
\end{lstlisting}


\section*{Задание 2}
\addcontentsline{toc}{section}{Задание 2}
Написать функцию, которая выбирает из заданного списка только те числа, которые больше 1 и меньше 10.

\begin{lstlisting}
(defun selector_func (lst a b)
	(remove nil 
		(mapcar (lambda (el) 
			(and 
				(numberp el) 
				(> el a) 
				(< el b) el)
				) lst
		)
	)
)

\end{lstlisting}
mapcar возвращает исходный список, в котором все непрошедшие проверку элементы заменены на nil. Эти элементы удаляются с помощью remove


\section*{Задание 3}
\addcontentsline{toc}{section}{Задание 3}
Почему так реализовано reduce, в чем причина?
\begin{lstlisting}
	(reduce #'+ ()) -> 0
\end{lstlisting}

Причина в том, что reduce должен корректно обрабатывать поданный ему список любого размера, поэтому когда список пуст и аргумент :initial-value не задан,  возвращается значение функции-аргумента, которую вызвали без аргументов, а если аргумент :initial-value задан, то возвращается его значение.




\section*{Задание 4}
\addcontentsline{toc}{section}{Задание 4}
Пусть list-of-lists список, состоящий из списков. Написать функцию, которая вычисляет сумму длин всех элементов list-of-lists, т.е. например для аргумента ((1 2) (3 4)) -> 4.

\subsubsection*{Хвостовая рекурсия:}
\begin{lstlisting}[caption=Функция-обертка для вычисления суммарной длины списков]
(defun count-length (lst)
	(count-length-helper lst 0)
)
\end{lstlisting}
\textbf{lst} - входной список.
\begin{lstlisting}[caption=Функция для вычисления суммарной длины списков]
(defun count-length-helper (lst length)
	(cond	
		(
			(null lst) 
			length
		)
		(
			(listp (car lst)) 
			(count-length-helper 
			(cdr lst) 
			(count-length-helper (car lst) length)) 
		)
		(
			(count-length-helper (cdr lst) (+ length 1))
		)
	)
)
\end{lstlisting}
\textbf{lst} - входной список, \textbf{length} - длина списка.

Условием выхода из рекурсии является достижение конца списка(первый аргумент - nil) - возвращается второй аргумент, в котором накапливается суммарная длина списка. Если голова первого аргумента список, то осуществляется рекурсивный вызов текущей функции для хвоста первого аргумента и рекурсивного вызова, который осуществляется для головы первого аргумента и второго аргумента. Иначе осуществляется рекурсивный вызов текущей функции для хвоста списка и второго аргумента, увеличенного на 1.

\subsubsection*{Примеры работы:}
\begin{lstlisting}
	> (count-length nil)
	0
	> (count-length '(1))
	1
	> (count-length '(1 2 3))
	3
	> (count-length '((1 2) (3 4)))
	4
	> (count-length '((1 2 (3) 4) 5 (6)))
	6
\end{lstlisting}




\section*{Ответы на вопросы:}
\addcontentsline{toc}{section}{Ответы на вопросы}
\hspace*{-7mm} \textbf{1) Классификация рекурсивных функций}
\\Рекурсия — это ссылка на определяемый объект во время его определения.
Способы организации рекурсивных функций:
\begin{itemize}
	\item Хвостовая рекурсия
	
	Результат формируется не на выходе из рекурсии, а на входе в рекурсию, все действия выполняются до ухода на следующий шаг рекурсии.
	\begin{lstlisting}
	(defun fun (x)
		(cond	
			(end_test1	end_value1)
			...
			(end_testN	end_valueN)
			( (fun reduced_x) )
		) 
	)
	\end{lstlisting}
	\item Рекурсия по нескольким параметрам
	\begin{lstlisting}
	(defun fun (n x)
		(cond	
			(end_test	end_value)
			( t	(fun (reduced_n) (reduced_x))) 
		)
	)
	\end{lstlisting}
	\item Дополняемая рекурсия
	
	При обращении к рекурсивной функции используется дополнительная функция не в аргументе вызова, а вне его.
	\begin{lstlisting}
	(defun fun (x)
		(cond	
			(test end_value)
			(t (add_fun	add_value	(fun reduced_x)) )
		)
	)
	\end{lstlisting}
	
	\item Множественная рекурсия
	
	На одной ветке происходит сразу несколько рекурсивных вызовов. Количество условий выхода также может зависеть от задачи.
	\begin{lstlisting}
	(defun fun (x)
		(cond	
			(test end_val)
			( t (combine	(fun changed1_x)
				(fun changed2_x))
			)
		)
	)
	\end{lstlisting}
\end{itemize}



\end{document}