\documentclass[a4paper, 12pt]{article}

%\usepackage{cmap}
\usepackage[T2A]{fontenc}
\usepackage[utf8]{inputenc}
\usepackage[english, russian]{babel}
\usepackage{graphicx}
\usepackage[top=1in, bottom=1in, left=3.2cm, right=2.6cm]{geometry}
\graphicspath{./}
\usepackage{biblatex}
\addbibresource{lib.bib}
\linespread{1.5}
\usepackage{ragged2e}
\justifying
\usepackage{listings}
\usepackage{color}


\begin{document}
	
\begin{titlepage}
	\fontsize{12pt}{12pt}\selectfont
	\begin{figure}[t!]
		\centering
		\includegraphics[scale=0.8]{bmstu}
	\end{figure}
	
	\noindent\rule{15cm}{3pt}
	\newline\newline
	\noindent 
	ФАКУЛЬТЕТ 
	\underline{«Информатика и системы управления»} \newline\newline
	
	\noindent КАФЕДРА \underline{«Программное обеспечение ЭВМ и информационные технологии»}\newline\newline\newline\newline\newline
	
	\centering {\Large Отчет по лабораторной работе № 8}
	\vspace{4mm}
	
	\centering {\Large По курсу: "Функциональное и логическое программирование"
		\vspace{8mm}	
		
		}
	\vspace{8mm}
	
	
	\begin{flushright}
		{\small	Студент:\\ Турсунов Жасурбек Рустамович \\ Группа: ИУ7-66Б
			\vspace{3mm}
			\\Преподователи: \\ Толпинская Наталья Борисовна \\ Строганов Юрий Владимирович}
	\end{flushright}
	
	\begin{center}
		\vfill
		Москва, \the\year
		~г.
	\end{center}
\end{titlepage}

\tableofcontents
\clearpage
\newpage

\textbf{Цель работы:} приобрести навыки использования функционалов.
\\ \hspace*{5mm} \textbf{Задачи работы:} описать и сравнить работу функционалов: apply, funcall, mapcar, maplist


\section*{Введение}
\addcontentsline{toc}{section}{Введение}

\hspace*{5mm} Для организации многократных вычислений в Lisp могут быть использованы функционалы - функции, которые особым образом обрабатывают свои аргументы. Функционалы - это  функции более высокого порядка, так как они в качестве своего первого аргумента принимают функциональный объект - функцию, имеющую имя(глобально определенную функцию), или функцию, не имеющею имени(локально определенную функцию). При использовании функционального объекта должно быть использовано замыкание контекста функции, которым обеспечивается связывание свободных переменных со значениями. В Lisp используются применяющие и отображающие функционалы, являющиеся предикатами, функционалы, использующие предикаты в качестве функционального объекта
\clearpage
\newpage


\definecolor{codegreen}{rgb}{0,0.6,0}
\definecolor{codegray}{rgb}{0.5,0.5,0.5}
\definecolor{codepurple}{rgb}{0.58,0,0.82}
\definecolor{backcolour}{rgb}{0.95,0.95,0.92}

\lstdefinestyle{mystyle}{
	backgroundcolor=\color{backcolour},   
	commentstyle=\color{codegreen},
	keywordstyle=\color{magenta},
	numberstyle=\tiny\color{codegray},
	stringstyle=\color{codepurple},
	basicstyle=\ttfamily\footnotesize,
	breakatwhitespace=false,         
	breaklines=false,                 
	captionpos=b,                    
	keepspaces=true,                 
	numbers=left,                    
	numbersep=5pt,                  
	showspaces=false,                
	showstringspaces=false,
	showtabs=false,                  
	tabsize=4
}

\lstset{style=mystyle}

\section*{Задание 1}
\addcontentsline{toc}{section}{Задание 1}
Напишите функцию, которая умножает на заданное число-аргумент все числа из заданного списка-аргумента, когда
\begin{enumerate}
	\item[а)] все элеметны списка - числа,
	\item[б)] элементы списка - любые объекты.\\
\end{enumerate}

\begin{lstlisting}[caption=Функция умножения для списка из чисел]
(defun mult_num (lst k)
	(mapcar 
		#'(lambda (el) (* el k))
		lst
	)
)
\end{lstlisting}
\textbf{lst} - входной список, \textbf{k} - число, на которое выполняется умножение.

С помощью mapcar осуществляется формирование списка, состоящего из элементов списка-аргумента, умноженных на число-аргумент.

\begin{lstlisting}[caption=Функция умножения для списка из любых объектов]
(defun mult_all (lst k)
	(mapcar #'(lambda (el)
		(cond
			(
				(numberp el)
				(* el k)
			)
			(
				(atom el)
				el
			)
			(
				(mult_all el k)
			)
		)
	  ) lst
	)
)
\end{lstlisting}
\textbf{lst} - входной список, \textbf{k} - число, на которое выполняется умножение.

С помощью mapcar существляется проход по всему списку и проверка типа каждого элемента: 
\begin{itemize}
	\item если он является числом, то производится умножение на число-аргумент функции и возвращается результат;
	\item если он является атомом, то он возвращается;
	\item если он является списком, то к нему рекурсивно применяется текущая функция.
\end{itemize}

\subsubsection*{Примеры работы:}
\begin{lstlisting}
	> (mult_num nil 1)
	NIL
	> (mult_num '(1) 3)
	(3)
	> (mult_num '(1 2 3 4 5) 3)
	(3 6 9 12 15)
	> (mult_all nil 2)
	NIL
	> (mult_all '(1) 2)
	(2)
	> (mult_all '(1 2 3) 3)
	(3 6 9)
	> (mult_all '((1 2) 3 (4 (5))) 3)
	((3 6) 9 (12 (15)))
	> (mult_all '(1 2 (a)) 3)
	(3 6 (A))
\end{lstlisting}
\section*{Задание 2}
\addcontentsline{toc}{section}{Задание 2}
Напишите функцию, select-between, которая из списка-аргумента, содержащего только числа, выбирает только те, которые расположены между двумя указанными границами-аргументами и возвращает их в виде списка.

\begin{lstlisting}[caption=Функция выборки из списка по границам]
(defun select-between (lst a b)
	(remove-if	#'(lambda (x) 
		(and 
			(or (< x a)(> x b)) 
			(or (> x a)(< x b))
		)
	  )lst
	)
)
\end{lstlisting}
\textbf{lst} -  входной список, \textbf{a}, \textbf{b}  -  границы выбора.

С помощью функционала remove-if осуществляется проход по всему списку-аргументу и удаление тех элементов списка, которые не входят в заданные границы-аргументы.

\subsubsection*{Примеры работы:}
\begin{lstlisting}
	> (select-between nil 1 2)
	NIL
	> (select-between '(1) 0 2)
	(1)
	> (select-between '(1 2 3) 4 5)
	NIL
	> (select-between '(1 2 3 4 5) 2 4)
	(2 3 4)
	> (select-between '(1 2 3 4 5) 4 2)
	(2 3 4)
\end{lstlisting}

\section*{Задание 3}
\addcontentsline{toc}{section}{Задание 3}
Напишите функцию, которая уменьшает на 10 все числа из списка аргумента этой функции.

\subsubsection*{Реализация с помощью функционалов и рекурсии:}
\begin{lstlisting}[caption=Функция уменьшения всех чисел смиска на 10]
(defun dec_all (lst)
	(mapcar #'(lambda (el)
		(cond
			(
				(numberp el)
				(- el 10)
			)
			(
				(atom el)
				el
			)
			(
				(dec_all el)
			)
		)
	   ) lst
	)
)
\end{lstlisting}
\textbf{lst} - список.

В реализации осуществляется проход по всему списку, проверка типа каждого элемента: 
\begin{itemize}
	\item если он является числом, то из него вычитается 10 и выделяется списочная ячейка; 
	\item если он является атомом, то под него выделяется списочная ячейка;
	\item если он является списком, то к нему рекурсивно применяется текущая функция.
\end{itemize}


\section*{Задание 4}
\addcontentsline{toc}{section}{Задание 4}
Написать функцию, которая возвращает первый аргумент списка-аргумента, который сам является непустым списком.

\subsubsection*{Реализация с помощью рекурсии:}
\begin{lstlisting}[caption=Функция поиска первого непустого списка]
(defun get_first_list (lst)
	(cond	
		(
			(and	
				(listp (car lst))
				(not (null (car lst)))
			)
			(car lst)
		)
		(
			(get_first_list (cdr lst))
		)
	)
)
\end{lstlisting}
\textbf{lst} - список.

Реализация с помощью рекурсии эффективнее и по памяти, и по скорости, так как не создает вспомогательных списков и возвращает непустой список сразу же, как он встретится.

\subsubsection*{Примеры работы:}
\begin{lstlisting}
	> (get_first_list '(1 2 (3) (4 5 (6))))
	(3)
	> (get_first_list '(((1 2) 3) 4 5))
	((1 2) 3)
	> (get_first_list '(nil 2 (nil (5)) 6))
	(NIL (5))
\end{lstlisting}



\section*{Задание 5}
\addcontentsline{toc}{section}{Задание 5}
Что будет результатом (mapcar 'вектор '(570-40-8))? 
\\ Результат: (\#(|570-40-8|))



\section*{Ответы на вопросы:}
\addcontentsline{toc}{section}{Ответы на вопросы}
\hspace*{-7mm} \textbf{1) Варианты использования функционалов}
\\ Функционалы используются для вычисления одной функции с разными входными значениями. Например, для применения какой-то операции ко всему списку. Некоторые функционалы используют повторные вычисления, что позволяет организовывать накие вычисления на этих функционалах (например, mapcar)





\end{document}