\documentclass[a4paper, 12pt]{article}

%\usepackage{cmap}
\usepackage[T2A]{fontenc}
\usepackage[utf8]{inputenc}
\usepackage[english, russian]{babel}
\usepackage{graphicx}
\usepackage[top=1in, bottom=1in, left=3.2cm, right=2.6cm]{geometry}
\graphicspath{./}
\usepackage{biblatex}
\addbibresource{lib.bib}
\linespread{1.5}
\usepackage{ragged2e}
\justifying
\usepackage{listings}
\usepackage{color}


\begin{document}
	
\begin{titlepage}
	\fontsize{12pt}{12pt}\selectfont
	\begin{figure}[t!]
		\centering
		\includegraphics[scale=0.8]{bmstu}
	\end{figure}
	
	\noindent\rule{15cm}{3pt}
	\newline\newline
	\noindent 
	ФАКУЛЬТЕТ 
	\underline{«Информатика и системы управления»} \newline\newline
	
	\noindent КАФЕДРА \underline{«Программное обеспечение ЭВМ и информационные технологии»}\newline\newline\newline\newline\newline
	
	\centering {\Large Отчет по лабораторной работе № 6}
	\vspace{4mm}
	
	\centering {\Large По курсу: "Функциональное и логическое программирование"
		\vspace{8mm}	
		
		}
	\vspace{8mm}
	
	
	\begin{flushright}
		{\small	Студент:\\ Турсунов Жасурбек Рустамович \\ Группа: ИУ7-66Б
			\vspace{3mm}
			\\Преподователи: \\ Толпинская Наталья Борисовна \\ Строганов Юрий Владимирович}
	\end{flushright}
	
	\begin{center}
		\vfill
		Москва, \the\year
		~г.
	\end{center}
\end{titlepage}

\tableofcontents
\clearpage
\newpage

\textbf{Цель работы:} приобрести навыки работы с управляющими структурами Lisp.
\\ \hspace*{5mm} \textbf{Задачи работы:} изучить работу функций с произвольным количеством аргументов, функций разрушающих и неразрушающих структуру исходных аргументов.


\section*{Введение}
\addcontentsline{toc}{section}{Введение}

\hspace*{5mm} Многие стандартные функции Lisp являются формами и реализуют особый способ работы со своими аргументами. К таким функциям относятся функции, позволяющие работать с произвольным количеством аргументов: and, or, append, или особым образом обрабатывающее свои аргументы: функции cond, if, append, remove, reverse, substitute.
\\ \hspace*{5mm} Если на вход функции подается структура данных(списков), то возникает вопрос: сохранится ли возможность в дальнейшем работать с исходными струтурами, или они изменятся в процессе реализации функции. В Lisp существуют функции, использующие списки в качестве аргументов и разрушающие структуру исходных аргументов при этом часть из них позволяет использовать произвольное количество аргументов, а часть нет.
\clearpage
\newpage


\definecolor{codegreen}{rgb}{0,0.6,0}
\definecolor{codegray}{rgb}{0.5,0.5,0.5}
\definecolor{codepurple}{rgb}{0.58,0,0.82}
\definecolor{backcolour}{rgb}{0.95,0.95,0.92}

\lstdefinestyle{mystyle}{
	backgroundcolor=\color{backcolour},   
	commentstyle=\color{codegreen},
	keywordstyle=\color{magenta},
	numberstyle=\tiny\color{codegray},
	stringstyle=\color{codepurple},
	basicstyle=\ttfamily\footnotesize,
	breakatwhitespace=false,         
	breaklines=false,                 
	captionpos=b,                    
	keepspaces=true,                 
	numbers=left,                    
	numbersep=5pt,                  
	showspaces=false,                
	showstringspaces=false,
	showtabs=false,                  
	tabsize=4
}

\lstset{style=mystyle}

\section*{Задание 1}
\addcontentsline{toc}{section}{Задание 1}
Чем принципиально отличаются функции CONS, LIST, APPEND?\\
CONS является базовой функцией Lisp, LIST и APPEND реализованы через CONS. CONS является чистой математической функцией, принимающей 2 параметра. Она создает списочную ячейку, CAR-указатель которой указывает на 1-й аргумент, а CDR-указатель - на 2-й аргумент.

LIST является формой, так как принимает произвольное число аргументов, возвращая список, состоящий из переданных ему аргументов.

APPEND является формой, так как принимает произвольное число аргументов, которые должны быть списками, кроме последнего. Она возвращает точечную пару, CAR-указатель которой указывает на конкатенацию всех переданных ей аргументов, кроме последнего, а CDR-указатель - на последний аргумент. При этом структура исходных аргументов не разрушается, для этого происходит копирование всех аргументов, кроме последнего. 

Пусть

(setf lst1 '( a b)) (setf lst2 '( c d))

Каковы результаты вычисления следующих выражений?
\begin{itemize}
	\item (cons lst1 lst2)\\
	Результат: ((A B) C D)
	\item (list lst1 lst2)\\
	Результат: ((A B) (C D))
	\item (append lst1 lst2)\\
	Результат: (A B C D)\\
\end{itemize}

\section*{Задание 2}
\addcontentsline{toc}{section}{Задание 2}
Каковы результаты вычисления следующих выражений?
\begin{itemize}
	\item (reverse ())\\
	Результат: NIL
	\item (last ())\\
	Результат: NIL
	\item (reverse '(a))\\
	Результат: (A)
	\item (last '(a))\\
	Результат: (A)
	\item (reverse '((a b c)))\\
	Результат: ((A B C))
	\item (last '((a b c)))\\
	Результат: ((A B C))\\
\end{itemize}

\section*{Задание 3}
\addcontentsline{toc}{section}{Задание 3}
Написать, по крайней мере, 2 варианта функции, которая возвращает последний элемент своего списка-аргумента.

\begin{lstlisting}[caption=1-й вариант функции получения последнего элемента списка]
	(defun get_last (lst) 
		(car (last lst))
	)
\end{lstlisting}

\begin{lstlisting}[caption=2-й вариант функции получения последнего элемента списка]
	(defun get_last (lst) 
		(car (reverse lst))
	)
\end{lstlisting}
\clearpage
\newpage
\begin{lstlisting}[caption=3-й вариант функции получения последнего элемента списка]
	(defun get_last (lst) (cond 
		(
			(cdr lst)
			(get_last (cdr lst))
		)
		(
			(car lst)
		)
		)
	)
\end{lstlisting}

Пример:
\begin{lstlisting}
	> (get_last '(1 2 3 4))
	4
	> (get_last '((1 2) 3 4 (5 (6))))
	(5 (6))
	> (get_last ())
	NIL
\end{lstlisting}
\section*{Задание 4}
\addcontentsline{toc}{section}{Задание 4}
Написать, по крайней мере, 2 варианта функции, которая возвращает свой список-аргумент без последнего элемента.

\begin{lstlisting}[caption=1-й вариант функции получения спиcка без последнего элемента]
	(defun remove_last (lst) 
		(reverse (cdr (reverse lst)))
	)
\end{lstlisting}

\begin{lstlisting}[caption=2-й вариант функции получения спиcка без последнего элемента]
	(defun remove_last (lst) (cond 
		(
			(cdr lst) 
			(cons 
				(car lst) 
				(remove_last (cdr lst))
			)
		)
	))
\end{lstlisting}

Пример:
\begin{lstlisting}
	> (remove_last '(1 2 3 4))
	(1 2 3)
	> (remove_last '((1 2) 3 4 (5 (6))))
	((1 2) 3 4)
	> (remove_last ())
	NIL
\end{lstlisting}

\section*{Задание 5}
\addcontentsline{toc}{section}{Задание 5}
Написать простой вариант игры в кости, в котором бросаются две правильные кости. Если сумма выпавших очков равна 7 или 11 - выигрыш, если выпало (1,1) или (6,6) - игрок получает право снова бросить кости, во всех остальных случаях ход переходит ко второму игроку, но запоминается сумма выпавших очков. Если второй игрок не выигрывает абсолютно, то выигрывает тот игрок, у которого больше очков. Результат игры и значения выпавших костей выводить на экран с помощью функции print.
\begin{lstlisting}[caption=Функция подбрасывания костей]
	(defun roll_dice () 
		(cons 
			(+ (random 6) 1) 
			(+ (random 6) 1)
		)
	)
\end{lstlisting}

\begin{lstlisting}[caption=Функция проверки переподбрасывания костей]
	(defun check_reroll (pair) 
		(or 
			(= (car pair) (cdr pair) 1) 
			(= (car pair) (cdr pair) 6)
		)
	)
\end{lstlisting}

\begin{lstlisting}[caption=Функция суммирования чисел в точечной паре]
	(defun sum_pair (pair) 
		(+ (car pair) (cdr pair))
	)
\end{lstlisting}

\begin{lstlisting}[caption=Функция проверки моментального выигрыша]
	(defun check_win (pair) 
		(or 
			(= (sum_pair pair) 7)
			(= (sum_pair pair) 11)
		)
	)
\end{lstlisting}

\begin{lstlisting}[caption=Функция получения конечного результата подбрасывания]				
	(defun get_player_result () 
		(let ((pair (toss_dice))) 
			(cond
				(
					(check_reroll pair)
					(format T "~%Got ~A, rerolling: " pair) (get_player_result)
				)
				(pair)
			)
		)
	)
\end{lstlisting}
\clearpage
\newpage
\begin{lstlisting}[caption=Функция игры в кости]
(defun play_dice () (princ "Player 1: ")
  (let ((player1 (get_player_result)))
    (princ player1)
	(terpri)
	(cond 
	  (
		(check_win player1)
		(princ "Player 1 wins")
	  )
	  (		
		(princ "Player 2: ")
	    (let ((player2 (get_player_result)))
		  (princ player2)
		  (terpri)
		  (cond 
			(
	          (check_win player2)
	          (princ "Player 2 wins")
	
	        )
	        (
	          (cond 
	            (
	              (= (sum_pair player1) (sum_pair player2))
				  (princ "Draw")
				)
				(
			      (> (sum_pair player1) (sum_pair player2))
				  (princ "Player 1 wins")
				)
				(
				  (princ "Player 2 wins")
				)	
			  )
		    )
		  )
		)
	  )
	)
  )
)
\end{lstlisting}

Пример:
\begin{lstlisting}
	> (play_dice)
	Player 1: (5 . 1)
	Player 2: (5 . 3)
	Player 2 wins
	> (play_dice)
	Player 1: (6 . 4)
	Player 2: (2 . 5)
	Player 2 wins
	> (play_dice)
	Player 1: (4 . 1)
	Player 2: (3 . 2)
	Draw
	> (play_dice)
	Player 1: (3 . 4)
	Player 1 wins
	> (play_dice)
	Player 1: 
	Got (1 . 1), rerolling: (4 . 6)
	Player 2: (2 . 3)
	Player 1 wins
\end{lstlisting}

\section*{Ответы на вопросы:}
\addcontentsline{toc}{section}{Ответы на вопросы}
\hspace*{-7mm} \textbf{1) Отличие в работе функций CONS, LIST, APPEND:}
CONS является базовой функцией Lisp, LIST и APPEND реализованы через CONS. CONS является чистой математической функцией, принимающей 2 параметра. Она создает списочную ячейку, CAR-указатель которой указывает на 1-й аргумент, а CDR-указатель - на 2-й аргумент.

LIST является формой, так как принимает произвольное число аргументов, возвращая список, состоящий из переданных ему аргументов.

APPEND является формой, так как принимает произвольное число аргументов, которые должны быть списками, кроме последнего. Она возвращает точечную пару, CAR-указатель которой указывает на конкатенацию всех переданных ей аргументов, кроме последнего, а CDR-указатель - на последний аргумент. При этом структура исходных аргументов не разрушается, для этого происходит копирование всех аргументов, кроме последнего.

\hspace*{-13mm} \textbf{2) Структурно разрушающие и не разрушающие структуру списка функции:}
\\ 



\end{document}